\usepackage{xcolor}
\usepackage{colortbl}
\usepackage{rotating}
% Table generated by Excel2LaTeX from sheet 'Hoja1'
\begin{tabular}{|c|c|c|c|c|c|c|c|}
\hline
  \multicolumn{ 5}{|c|}{{\bf Tabla de materias y actividades}} & \multicolumn{ 3}{|c|}{{\bf Ingenier�a Mecatr�nica 7�A}} \\
\hline
\multicolumn{ 1}{|c|}{{\bf Integrantes}} & \multicolumn{ 7}{|c|}{C�sar Omar Alvarado Contreras. Jonathan Fonseca Camarena. Marcos Manzo Torres. Eduardo Robles V�zquez. V�ctor Gabriel Tapia Casillas.} \\

\multicolumn{ 1}{|c|}{{\bf }} &                                                                  \multicolumn{ 7}{|c|}{} \\
\hline
\multicolumn{ 2}{|c|}{{\bf Materia}} & \multicolumn{ 3}{|c|}{{\bf Actividades}} & \multicolumn{ 3}{|c|}{{\bf Maestro}} \\
\hline
\multicolumn{ 2}{|c|}{Administraci�n de proyectos de ingenier�a} & \multicolumn{ 3}{|c|}{Capacidad de administrar el tiempo, recursos y actividades a desarrollar para la realizaci�n del proyecto.} & \multicolumn{ 3}{|c|}{Miguel Alberto Martinez Molina} \\
\hline
\multicolumn{ 2}{|c|}{} &              \multicolumn{ 3}{|c|}{} &              \multicolumn{ 3}{|c|}{} \\
\hline
\multicolumn{ 2}{|c|}{} &              \multicolumn{ 3}{|c|}{} &              \multicolumn{ 3}{|c|}{} \\
\hline
\multicolumn{ 2}{|c|}{Cinem�tica de robots} & \multicolumn{ 3}{|c|}{Proveer de aspectos t�cnicos de los distintos tipos de robots, adem�s de su funcionamiento, programaci�n, c�lculos y caracter�sticas. } & \multicolumn{ 3}{|c|}{Carlos Enrique Mor�n Garabito} \\
\hline
\multicolumn{ 2}{|c|}{} &              \multicolumn{ 3}{|c|}{} &              \multicolumn{ 3}{|c|}{} \\
\hline
\multicolumn{ 2}{|c|}{} &              \multicolumn{ 3}{|c|}{} &              \multicolumn{ 3}{|c|}{} \\
\hline
\multicolumn{ 2}{|c|}{} &              \multicolumn{ 3}{|c|}{} &              \multicolumn{ 3}{|c|}{} \\
\hline
\multicolumn{ 2}{|c|}{Dise�o y selecci�n de elementos mec�nicos} & \multicolumn{ 3}{|c|}{Brindar informaci�n de los distintos tipos de materiales a utilizar, sus ventajas y desventajas. Complementaci�n de aprendizaje del uso de softwares para el dise�o de planos y prototipos.} & \multicolumn{ 3}{|c|}{Norberto Garc�a Alvarez} \\
\hline
\multicolumn{ 2}{|c|}{} &              \multicolumn{ 3}{|c|}{} &              \multicolumn{ 3}{|c|}{} \\
\hline
\multicolumn{ 2}{|c|}{} &              \multicolumn{ 3}{|c|}{} &              \multicolumn{ 3}{|c|}{} \\
\hline
\multicolumn{ 2}{|c|}{} &              \multicolumn{ 3}{|c|}{} &              \multicolumn{ 3}{|c|}{} \\
\hline
\multicolumn{ 2}{|c|}{} &              \multicolumn{ 3}{|c|}{} &              \multicolumn{ 3}{|c|}{} \\
\hline
\multicolumn{ 2}{|c|}{} &              \multicolumn{ 3}{|c|}{} &              \multicolumn{ 3}{|c|}{} \\
\hline
\multicolumn{ 2}{|c|}{Ingl�s} & \multicolumn{ 3}{|c|}{Dotar de habilidades ling�isticas para la comprensi�n de diversos textos que puedan ser de ayuda, con la caracter�stica de encontrarse en ingl�s.} & \multicolumn{ 3}{|c|}{Mauro Ceballos Heredia} \\
\hline
\multicolumn{ 2}{|c|}{} &              \multicolumn{ 3}{|c|}{} &              \multicolumn{ 3}{|c|}{} \\
\hline
\multicolumn{ 2}{|c|}{} &              \multicolumn{ 3}{|c|}{} &              \multicolumn{ 3}{|c|}{} \\
\hline
\multicolumn{ 2}{|c|}{} &              \multicolumn{ 3}{|c|}{} &              \multicolumn{ 3}{|c|}{} \\
\hline
\multicolumn{ 2}{|c|}{} &              \multicolumn{ 3}{|c|}{} &              \multicolumn{ 3}{|c|}{} \\
\hline
\multicolumn{ 2}{|c|}{Modelado y simulaci�n de sistemas} & \multicolumn{ 3}{|c|}{Aprender a aplicar distintos sistemas y modelado matem�ticos para la simulaci�n y desarrollo del prototipo.} & \multicolumn{ 3}{|c|}{Rosa Mar�a Razo Cerda} \\
\hline
\multicolumn{ 2}{|c|}{} &              \multicolumn{ 3}{|c|}{} &              \multicolumn{ 3}{|c|}{} \\
\hline
\multicolumn{ 2}{|c|}{} &              \multicolumn{ 3}{|c|}{} &              \multicolumn{ 3}{|c|}{} \\
\hline
\multicolumn{ 2}{|c|}{} &              \multicolumn{ 3}{|c|}{} &              \multicolumn{ 3}{|c|}{} \\
\hline
\multicolumn{ 2}{|c|}{Termodin�mica} & \multicolumn{ 3}{|c|}{Obtener conocimientos acerca del comportamiento t�rmico del prototipo. Esto debido a que al ser un sistema mec�nico genera fricci�n entre sus partes y, por ende, calor. } & \multicolumn{ 3}{|c|}{Jos� Carlos D�az Nu�ez} \\
\hline
\multicolumn{ 2}{|c|}{} &              \multicolumn{ 3}{|c|}{} &              \multicolumn{ 3}{|c|}{} \\
\hline
\multicolumn{ 2}{|c|}{} &              \multicolumn{ 3}{|c|}{} &              \multicolumn{ 3}{|c|}{} \\
\hline
\multicolumn{ 2}{|c|}{} &              \multicolumn{ 3}{|c|}{} &              \multicolumn{ 3}{|c|}{} \\
\hline
\multicolumn{ 2}{|c|}{} &              \multicolumn{ 3}{|c|}{} &              \multicolumn{ 3}{|c|}{} \\
\hline
\end{tabular}  

